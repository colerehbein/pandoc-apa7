% options for packages loaded elsewhere
\PassOptionsToPackage{unicode=true}{hyperref}
\PassOptionsToPackage{hyphens}{url}

% apa6 mode and class options
\documentclass[,longtable]{apa6}

% for mode selection options
\usepackage{ifthen}

% setup mode ifs
\newif\ifmanmode
\newif\ifdocmode
\newif\ifjoumode
\ifthenelse{\equal{\string }{\string man}}{
    \manmodetrue
}{
    \ifthenelse{\equal{\string }{\string doc}}{
        \docmodetrue
    }{
        \ifthenelse{\equal{\string }{\string jou}}{
            \joumodetrue
        }{
% None
}}}

% man mode uses the endfloat package,
% this will try to set tables and figures at the end for longtables too
\ifmanmode
\DeclareDelayedFloatFlavor{longtable}{table}
\fi

% other packages
\usepackage{lmodern}
\usepackage{amsmath,amssymb}
\usepackage{ifxetex,ifluatex}
\usepackage{fixltx2e} % provides \textsubscript

% handle different types of tex engines
% if pdftex
\ifnum 0\ifxetex 1\fi\ifluatex 1\fi=0
  \usepackage[T1]{fontenc}
  \usepackage[utf8]{inputenc}
  \usepackage{textcomp} % provides euro and other symbols
% if luatex or xelatex
\else
  \usepackage{unicode-math}
  \defaultfontfeatures{Ligatures=TeX,Scale=MatchLowercase}
\fi

% other language options
\usepackage[american]{babel}
\usepackage{csquotes}
\usepackage{microtype}

% disable microtype protrusion for tt fonts
\UseMicrotypeSet[protrusion]{basicmath}

\usepackage{graphicx,grffile}
% Scale images if necessary, so that they will not overflow the page
% margins by default, and it is still possible to overwrite the defaults
% using explicit options in \includegraphics[width, height, ...]{}
\makeatletter
\def\maxwidth{\ifdim\Gin@nat@width>\linewidth\linewidth\else\Gin@nat@width\fi}
\def\maxheight{\ifdim\Gin@nat@height>\textheight\textheight\else\Gin@nat@height\fi}
\makeatother
\setkeys{Gin}{width=\maxwidth,height=\maxheight,keepaspectratio}

\usepackage{booktabs}
\usepackage{caption}
\usepackage{subcaption}

% Pandoc stuff
\let\tightlist\relax % empty pandoc tight list command

% Hyperlinks and other metadata in pdf
\usepackage{hyperref}
\hypersetup{
            colorlinks=false,
            linkcolor=Black,
            citecolor=Black,
            urlcolor=Black,
            pdfborder={0 0 0},
            breaklinks=true}
\urlstyle{same} % don't use monospace font for urls






\title{~}\shorttitle{~}\author{~}\affiliation{~}





\ifdocmode
\usepackage{longtable}
\fi

\ifjoumode
% Journal specific commands may go here:
% two column long table. has to be done at the end
\usepackage{longtable}
\makeatletter
\let\oldlt\longtable
\let\endoldlt\endlongtable
\def\longtable{\@ifnextchar[\longtable@i \longtable@ii}
\def\longtable@i[#1]{%

\begin{figure}[btph]\onecolumn \begin{minipage}{0.5\textwidth} \oldlt[#1]}
\def\longtable@ii{\begin{figure}[btph]\onecolumn \begin{minipage}{0.5\textwidth} \oldlt}
\def\endlongtable{%
\endoldlt
\end{minipage}

\twocolumn
\end{figure}}
\makeatother
\fi

% table caption width
\makeatletter
\newcommand\LastLTentrywidth{1em}
\newlength\longtablewidth
\setlength{\longtablewidth}{1in}
\newcommand\getlongtablewidth{%
\begingroup
\ifcsname LT@\roman{LT@tables}\endcsname
\global\longtablewidth=0pt
\renewcommand\LT@entry[2]{\global\advance\longtablewidth by ##2\relax\gdef\LastLTentrywidth{##2}}%
\@nameuse{LT@\roman{LT@tables}}%
\fi
\endgroup}


\begin{document}


\hypertarget{pandoc-apa}{%
\section{pandoc-apa}\label{pandoc-apa}}

A Sublime Text build system / Visual Studio Code task runner for writing
APA manuscripts with Pandoc formatted markdown

\hypertarget{installation}{%
\section{Installation}\label{installation}}

\hypertarget{pandoc}{%
\subsection{Pandoc}\label{pandoc}}

You need Pandoc to use the pandoc templates. You can find it here:
\url{https://pandoc.org/installing.html}

\hypertarget{latex-word}{%
\subsection{LaTex / Word}\label{latex-word}}

You'll need a LaTeX distribution to build the APA formatted pdf files
(e.g., MikTex or MacTex), or Microsoft Word to convert markdown to .docx
files.

\hypertarget{filters}{%
\subsection{Filters}\label{filters}}

Pandoc doesn't yet support cross referencing figures and tables. You'll
need a Python distribution to install two filters to be able to do this.
The filters are \texttt{pandoc-fignos} and \texttt{pandoc-tablenos}. You
can find links and installation instructions here
\url{https://github.com/tomduck/pandoc-fignos}

\hypertarget{usage}{%
\section{Usage}\label{usage}}

\hypertarget{sublime-text}{%
\subsection{Sublime Text}\label{sublime-text}}

Copy the \texttt{pandoc} and \texttt{sublime} directories from this
repository wherever your Packages folder is located in Sublime Text. For
example, if the contents are not already in a folder called
\texttt{pandoc-apa}, make a folder called \texttt{pandoc-apa} and copy
contents here if on Windows:
\texttt{C:\textbackslash{}Users\textbackslash{}username\textbackslash{}AppData\textbackslash{}Roaming\textbackslash{}Sublime\ Text\ 3\textbackslash{}Packages\textbackslash{}User\textbackslash{}pandoc-apa}.

\begin{itemize}
\item
  Press \texttt{ctrl\ +\ b} to build to pdf (or \texttt{cmd+b})
\item
  Press \texttt{ctrl\ +\ shift\ +\ b} to build to docx (or
  \texttt{cmd+shift+b})
\item
  You can also go to
  \texttt{Tools\ \textgreater{}\ Build\ System\ \textgreater{}\ Pandoc\ APA},
  then if you select \texttt{Tools\ \textgreater{}\ Build\ Width} you
  can see the different options. Default is .pdf.
  \texttt{Markdown\ to\ LaTeX} will convert a pandoc markdown document
  to a .tex file. \texttt{Run} will convert to .docx.
\item
  In a document, type \texttt{apayaml}, then press Tab. It will generate
  a YAML header. Or you can do this from
  \texttt{Tools\ \textgreater{}\ Snippets\ \textgreater{}\ "Insert\ Pandoc\ ..."}
\end{itemize}

\hypertarget{visual-studio-code}{%
\subsection{Visual Studio Code}\label{visual-studio-code}}

Copy the \texttt{.vscode} and \texttt{pandoc} directories to where your
main markdown file is located. Open the folder containing your markdown
file as a vscode workspace. Press \texttt{Ctrl+Shift+B} to run the build
task or press \texttt{F1} then type \texttt{task}, select \enquote{run
task}, then choose one of the Pandoc APA options.

Instead of copying the pandoc-apa contents for each project, you can now
create a workspace with multiple root folders in vs code. Make a
\texttt{.code-workspace} file where your main .md file is located with
something like this, where the second path is pointing to the pandoc-apa
directory:

\begin{verbatim}
{
    "folders": [
        {
            "path": "."
        },
        {
            "path": "../pandoc-apa"
        }
    ],
    "settings": {
        "files.exclude": {
            "example": true
        }
    }
}
\end{verbatim}

\hypertarget{pandoc-yaml-metadata-block}{%
\subsection{Pandoc YAML Metadata
Block}\label{pandoc-yaml-metadata-block}}

These templates makes heavy use of the pandoc metadata block at the
beginning of your document. I've also provided a snippet you can use to
insert the metadata and fill in what is needed (Tools \textgreater{}
Snippets). Most of the commands used in the \texttt{apa6} package are
fields that can be used in the YAML header. See documentation for more
details:
\url{http://mirror.hmc.edu/ctan/macros/latex/contrib/apa6/apa6.pdf}.
Fields used in this template are:

\begin{itemize}
\tightlist
\item
  \texttt{mode}

  \begin{itemize}
  \tightlist
  \item
    Set to \texttt{man}, \texttt{doc}, or \texttt{jou}. This is the only
    \emph{absolutely} necessary option to generate a file with no
    errors.
  \end{itemize}
\item
  \texttt{title}

  \begin{itemize}
  \tightlist
  \item
    Main title of the document. Consistent with the pandoc variable.
  \end{itemize}
\item
  \texttt{subtitle}

  \begin{itemize}
  \tightlist
  \item
    Running head title. Consistent with the pandoc variable.
  \end{itemize}
\item
  \texttt{author}

  \begin{itemize}
  \tightlist
  \item
    List of authors. Put multiple authors on one line to group by
    affiliation. Consistent with the pandoc variable.
  \end{itemize}
\item
  \texttt{institute}

  \begin{itemize}
  \tightlist
  \item
    List of affiliations. Won't work with docx. You have to add this
    manually.
  \end{itemize}
\item
  \texttt{twogroups}, \texttt{threegroups}, \ldots{} ,
  \texttt{sixgroups}

  \begin{itemize}
  \tightlist
  \item
    Set one of these to true if grouping authors by different
    affiliations, as in the twogroup YAML example. If anyone of these
    options are not set, it will use the standard
    \texttt{\textbackslash{}author\{\}} command.
  \end{itemize}
\item
  \texttt{authornote}

  \begin{itemize}
  \tightlist
  \item
    A note that will be placed at the bottom of the title page, or
    footnote if in \texttt{jou} mode.
  \end{itemize}
\item
  \texttt{bibliography}

  \begin{itemize}
  \tightlist
  \item
    Location of your bibtex references file, whatever pandoc-citeproc
    will read. Can add multiple fields for multiple files.
  \end{itemize}
\item
  \texttt{date}

  \begin{itemize}
  \tightlist
  \item
    Place anything typed here as you would in the
    \texttt{\textbackslash{}note\{\}} apa6 latex command.
  \end{itemize}
\item
  \texttt{keywords}

  \begin{itemize}
  \tightlist
  \item
    List of keywords that will show under the abstract. See YAML
    metadata example.
  \end{itemize}
\item
  \texttt{abstract}

  \begin{itemize}
  \tightlist
  \item
    The abstract text of the manuscript.
  \end{itemize}
\item
  \texttt{floatsintext} (optional)

  \begin{itemize}
  \tightlist
  \item
    If this field is present and set to \texttt{true}, it will keep
    figures and tables with text or at the end of the document.
  \end{itemize}
\item
  \texttt{classoption} (optional)

  \begin{itemize}
  \tightlist
  \item
    A list of options that the \texttt{apa6} package will understand.
    See link to manual above. Don't use the LaTeX options \texttt{jou},
    \texttt{man}, or \texttt{doc} here. Use the dedicated YAML field
    \texttt{mode} instead. If the field isn't used, it defaults to
    manuscript mode. Also the \texttt{longtable} option is already
    specified since it's necessary for pandoc to work with tables. Don't
    enter it twice. Some useful options may be \texttt{noextraspace},
    \texttt{draftfirst}, etc\ldots{}
  \end{itemize}
\item
  \texttt{joucommands} (optional)

  \begin{itemize}
  \tightlist
  \item
    If you are in journal mode and want to use the other journal
    commands, the commands are as follows (also see the link to the apa6
    manual for more details).
  \item
    \texttt{leftheader}
  \item
    \texttt{journal}
  \item
    \texttt{volume}
  \item
    \texttt{ccoppy}
  \item
    \texttt{copnum}
  \end{itemize}
\item
  \texttt{colorlinks} (optional)

  \begin{itemize}
  \tightlist
  \item
    Colorize links and citations.
  \end{itemize}
\end{itemize}

You can try additional fields not mentioned here:

\url{https://pandoc.org/MANUAL.html\#variables-set-by-pandoc}



\end{document}
